\documentclass{article}


\usepackage{\latexVars}
\usepackage{\tenxDir/latex/preamble}

\begin{document}

\title{\reportTitle \\
  \vspace{5mm}
  \large \projectName \\
}

\author{\reportAuthor}
\maketitle


\vspace{\fill}
\noindent
{\large \textbf{Sample}: \sample}

\vspace{3mm}
\noindent
{\large \textbf{Run specs}: \runDetails}

\vspace{3mm}
\noindent
{\large \textbf{Code}: \url{https://github.com/sansomlab/tenx}}



\clearpage


\tableofcontents

\clearpage


\section{Analysis parameters}

These should match those given in the introduction of the accompanying summary report.
\vspace{5mm}
The core of the data analysis was performed using \href{http://satijalab.org/seurat/}{Seurat} and \href{https://scanpy.readthedocs.io/}{scanpy}:

\begin{itemize}
\item The construction of the nearest neighbor graph, clustering and UMAP computation were performed using scanpy.
\item The differential expression analysis was performed using Seurat.
\item The geneset analysis was performed using \href{https://github.com/sansomlab/gsfisher}{gsfisher}
\item Please see \href{https://github.com/sansomlab/tenx}{https://github/sansomlab/tenx} for more details.
\end{itemize}

The key parameter choices used for this analysis were:

\begin{itemize}
\item The number of \reductionType{} components: \nPCs
\item The number of nearest neighbors: \nnK
\item The distance metric used for the nearest neighbor graph: \nnMetric
\item The resolution of the clustering: \resolution
\item The clustering algorithm: \clusteringAlgorithm
\item The differential expression test: \deTest
\end{itemize}



\subsection{Optional tasks}

This table summarises the status of the optional tasks. Tasks set to ``True'' were run.

\hspace*{1cm}

\input{task.summary.table.tex}

\clearpage

\clearpage

\section{Expression of genes of interest}
\input{\genelistsDir/plot.rdims.known.genes}
\clearpage


\section{Expression of de-novo identified cluster marker genes}

Here, we aim to visualise the expression of the ``best'' cluster marker genes. The genes shown are selected as follows:

\begin{enumerate}
\item Only strong marker genes passing the following filters were retained:
  \begin{itemize}
  \item BH adjusted p.value < 0.01
  \item absolute log fold change (cluster of interest vs all other cells) > 1
  \item mean expression level (in cluster of interest) > 2
  \item percent cells expressing (in cluster of interest) > 25
  \end{itemize}
\item The remaining strong marker genes were then ranked according to a heuristic score calculated as the geometric mean of (i) the 1 - adjusted p value, (ii) the log2 mean expression level (in the cluster of interest) and (iii) the absolute fold change.
  \end{enumerate}

\clearpage
\subsection{Positive cluster marker genes}

Up to \nPositiveMarkers{} top-scoring positive marker genes are shown for each cluster.

\input{\clusterMarkerRdimsPlotsDir/top.positive.cluster.markers}

\clearpage
\subsection{Negative cluster marker genes}

Up to \nNegativeMarkers{} top-scoring negative marker genes are shown for each cluster.

\input{\clusterMarkerRdimsPlotsDir/top.negative.cluster.markers}
