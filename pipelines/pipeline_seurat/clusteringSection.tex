\subsection{Selection of principal components}

The table shows the principal components that were retained for data visualisation and clustering.

\input{\clusterDir/selected.principal.components.tex}


\subsection{Assesment of the tSNE perplexity hyperparameter}

\input{\tsneDir/tSNE.perplexity}
\clearpage

\subsection{tSNE parameters}

The parameters chosen for the subsequent tSNE plots are:

\begin{itemize}
\item Perplexity: \tSNEPerplexity
\item Maximum number of iterations: \tSNEMaxIter
\item Use fast approximation: \tSNEFast
\end{itemize}

\subsection{tSNE plots colored by factors of interest (e.g. by cluster)}

\input{\tsneDir/plot.rdims.tsne.factor}

\subsection{UMAP plots colored by factors of interest}

\input{\umapDir/plot.rdims.umap.factor}

\subsection{Diffusion map}

\input{\diffmapDir/plot.diffusion.map}

\subsection{Breakdown of cell numbers by factors of interest (e.g. by cluster)}

\input{\groupNumbersDir/number.plots}

\subsection{Exploring the similarities between the clusters}

The distances between the clusters was assessed using the ``BuildClusterTree'' function in the Seurat package, which ``constructs a phylogenetic tree relating the ``average'' cell from each identity class''.

\begin{figure}[H]
\includegraphics[width=1.0\textwidth,height=0.9\textheight,keepaspectratio]{{{\clusterDir/cluster.dendrogram}}}
\caption{Visualisation of inter-cluster distances (cluster average, gene-based)}
\end{figure}

An alternative distance metric is the median pairwise pearson correlation between the cells of each cluster in pca space.

\begin{figure}[H]
\includegraphics[width=1.0\textwidth,height=0.9\textheight,keepaspectratio]{{{\clusterDir/cluster.correlation.dendrogram}}}
\caption{Visualisation of inter-cluster distances (median pairwise pearson correlation in pca space)}
\end{figure}

The median pairwise pearson correlations between the cells of each cluster (in pca space) are shown in the table below.

\input{\clusterDir/cluster.pairwise.correlations.tex}

\subsection{Comparison of different clustering resolutions}

The clustree algorithm is used to compare the different clustering resolutions.

\begin{figure}[H]
\includegraphics[width=1.0\textwidth,height=0.9\textheight,keepaspectratio]{{{\clusterDir/clustree}}}
\caption{The relationships between the clusters identified at different resolutions}
\end{figure}
